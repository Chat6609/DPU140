\documentclass[16 pt]{amsart}
\usepackage{amscd,amsmath,amsthm,amssymb}
\usepackage{enumerate,varioref}
\usepackage{epsfig}
\usepackage{graphicx}
\usepackage{mathtools}
\newtheorem{thm}{Theorem}
\newtheorem{cor}[thm]{Corollary}
\newtheorem{lem}[thm]{Lemma}
\newtheorem{prop}[thm]{Proposition}
\theoremstyle{definition}
\newtheorem{defn}[thm]{Definition}
\theoremstyle{remark}
\newtheorem{ex}[thm]{Example}
\newtheorem{rem}[thm]{Remark}
\numberwithin{equation}{subsection}
\newcommand{\R}{\mathbb{R}}
\newcommand{\Z}{\mathbb{Z}}
\newcommand{\C}{\mathbb{C}}
\newcommand{\Q}{\mathbb{Q}}
\newcommand{\lh}{\lim_{h\rightarrow 0}}
\begin{document}

\title{Homework 2 Maths 140 Winter 2017 \\ DePaul University\\Dr. Alexander}
\maketitle

\section{Introduction}

In this homework we will explore the beginnings of predicate logic and universal statements.  We will also explore the importance of domains in universal statements.


\section{Problems}

Problem 1. (a) Negate the following statement:

\[
\forall x\in \R, \text{ If } e^x > 1 \text{ then } x > 0.
\]

(b) The statement in (a) is true.  Therefore the negation is false.  Why does $e^{\ln(1/2)} = \frac{1}{2}$ not violate the above statement?\\

(c) Negate the following statement:

\[
\exists x\in \R \text{ so that } e^x < 0.
\]

(d) The statement in (c) is false, therefore its negation is true.  Give a counterexample to show that the statement is, in fact, false.

\newpage

Problem 2.  Consider the predicate $P(x)$ on three different domains, integers, rationals, and reals.

\begin{itemize}
\item[Integer:] $\forall x\in \Z, P(x)$\\
\item[Rational:] $\forall x\in \Q, P(x)$\\
\item[Real:] $\forall x\in \R, P(x)$\\
\end{itemize}

In this case, the domain changes the truth of the statement.  For example, $P(x) = ``x^2-x \ge 0"$ is true when $x$ is an integer, but otherwise false, since $(1/2)^2-(1/2) = -(1/4)$.  

(a) Find a predicate $P(x)$ so that the first two statements are true, but the final statement is false.\\

(b) Find a predicate $P(x)$ so that the final two statements are true, but the first statement is false.\\

(c) Find a predicate $P(x)$ so that the first and third statements are true, but the second statement is false.\\

Hint: Parts (a) and (b) are easy.  Part (c) will require a lot more thought.

\newpage

Problem 3. Find the truth set $D$ for the following statement. Let $a,b,c$ be positive integers where $a<b<c$.

\[
\forall x\in D, (x-a)(x-b)(x-c)>0.
\]  

\newpage

Problem 4. Consider the following argument form:

\begin{itemize}
\item $\forall k\in \mathbb{N}_0, P(k) \rightarrow P(k+1)$\\
\item $P(0)$\\
\item $\therefore \forall n\in \mathbb{N}_0, P(n)$
\end{itemize}

(a) Is this a valid argument form?\\

(b) Use this argument form to check the soundness of the following statement:
\[
1 + 2 + 4 + \cdots + 2^k = 2^{k+1} - 1.
\]

\end{document}