\documentclass[16 pt]{amsart}
\usepackage{amscd,amsmath,amsthm,amssymb}
\usepackage{enumerate,varioref}
\usepackage{epsfig}
\usepackage{graphicx}
\usepackage{mathtools}
\newtheorem{thm}{Theorem}
\newtheorem{cor}[thm]{Corollary}
\newtheorem{lem}[thm]{Lemma}
\newtheorem{prop}[thm]{Proposition}
\theoremstyle{definition}
\newtheorem{defn}[thm]{Definition}
\theoremstyle{remark}
\newtheorem{ex}[thm]{Example}
\newtheorem{rem}[thm]{Remark}
\numberwithin{equation}{subsection}
\newcommand{\R}{\mathbb{R}}
\newcommand{\Z}{\mathbb{Z}}
\newcommand{\C}{\mathbb{C}}
\newcommand{\Q}{\mathbb{Q}}
\newcommand{\lh}{\lim_{h\rightarrow 0}}
\begin{document}

\title{Midterm Maths 140 Autumn 2016 \\ DePaul University\\Dr. Alexander}
\maketitle
You have 90 minutes to complete this exam.  Calculators are allowed, but no other electronic devices are permitted.  Please write all your answers in complete, legible sentences, and show all your work to receive full credit.  There are seven (7) problems here.  You may choose to do any six (6) of them.  
\vspace{1in}


%table
\begin{center}
  \begin{tabular}{ c | c }
    Problem & Score\\
    \hline
    &\\
    1&\\
    &\\
    2&\\
    &\\
    3&\\
    &\\
    4&\\
    &\\
    5&\\
    &\\
    6&\\
    &\\
    7&\\
    &\\
    Bonus&\\
    &\\
    \hline 
    &\\    
    Total& 
 \end{tabular}
\end{center}

\newpage 
Problem 1. Give the proper negation of the statement:
\[
\text{``If } x+y > 5 \text{ then } x^2+y^2 > 25"
\]

\vspace{1in}

Solution: Recall that the negation of a conditional is
\[
\sim(p\rightarrow q) \equiv p\wedge \sim q
\]

So our proper negation is

\[
x+y>5 \text{ and } x^2+y^2 \le 25
\]

Notice that the negation of greater is not ``less" it's ``less than or equal."


\newpage
Problem 2. Is the following argument valid?  Justify your answer.\\


\begin{itemize}
\item[] $p\rightarrow r$\\
\item[] $q \rightarrow p $\\
\item[] $s \vee q$\\
\item[] $\sim s$
\item[] $\therefore r$
\end{itemize}


\vspace{1in}

Solution:

Let's first look at arguments 1 and 2.  Using the argument form of transitivity we see

\[
q\rightarrow p
\]

In arguments 3 and 4 we see
\[
(s\vee q) \text{ and } \sim s \text{ therefore } q
\]


So we finally have the classic argument form (after two reductions)

\begin{itemize}
\item[] $q\rightarrow r$\\
\item[] $q$\\
\item[] $\therefore r$
\end{itemize}

Which is valid.

\newpage

Problem 3. Write the proper negation of the following statement:

\[
\forall x,y,z\in\Z \text{, If } x < 4 \text{ and } y\ge 4 \text{ then } z \ne 16
\]

\vspace{1in}

Solution: We've seen the negation of a conditional in problem 1.  We also know that the negation of ``every" is ``there is one that fails"

So the proper negation is


\[
\exists x,y,z\in\Z \text{so that } x < 4 \text{ and } y\ge 4 \text{ and } z = 16
\]



\newpage

Problem 4. Prove the following statement or give a counterexample:

\[
\text{``The sum of five odd integers is an odd integer."}
\]

\vspace{1in}

Solution: Let $n_1,n_2,n_3,n_4,n_5$ be five odd integers.  Then there are five integers $j_1,j_2,j_3,j_4,j_5$ so that

\[
\forall i \in \{1,2,3,4,5\} n_i = 2j_i+1
\]


So now we sum these five integers

\[
n_1+n_2+n_3+n_4+n_5 = (2j_1+1) + (2j_2+1) + (2j_3+1) + (2j_4+1) + (2j_5+1)
\]

Factoring out the 2 we have
\[
n_1+\cdots + n_5 = 2(j_1+j_2+j_3+j_4+j_5) + 5
\]

We want the sum to be of the form $2t+1$ where $t$ is some integer.  In this case we have a $2t+5$ Notice, however, that $5= 2k+1$ where $k=2$ and so we can pull another 2 into the parantheses

\[
n_1+\cdots + n_5 = 2(j_1+j_2+j_3+j_4+j_5+2)+1
\]

We know that the sum in the parantheses is an integer because integers are closed under addition.
Thus the sum of five odd integers is odd.


\newpage

Problem 5. Prove the following or give a counterexample:

\[
\text{For every rational number } r \text{ there exists a nonzero integer } n \text{ so that } r^n \in \Z
\]


\vspace{1in}

Solution: We know this to be false.  Consider the following counterexample.

\[
r= \frac{2}{3}
\]

Notice that $2^j \ne 3^k$ no matter which $j$ or $k$ we pick.  So
\[
\forall n>0, 0<\left(\frac{2}{3}\right)^n <1 
\] 
and when $n<0$ (let $m=-n$)

\[
\forall m>0, \left(\frac{3}{2}\right)^m \notin \Z
\]



\newpage

Problem 6. Prove the following or give a counterexample:
\[
\text{``If } n\in\Z \text{ and } n^5 \text{ is odd, then  } n \text{ is odd."} 
\]

\vspace{1in}

Let's prove this by contrapositive.  The contrapositive of the statement is
\[
\text{ If } n \text{ is an even integer, then } n^5 \text{ is an even integer.}
\]

We can prove this very easily. Suppose $n$ is even, then there is some $k\in\Z$ so that $n=2k$.

Then $n^5 = (2k)^5 = 32k^5 = 2(16k^5)$ which is even.

So by the equivalence of the contrapositive we see that if $n$ is an integer and $n^5$ is odd, then $n$ itself is odd.


\newpage

Problem 7. Is the following statement true or false?  If true, prove it, if false, give a counterexample.
\[
\text{``If } r\in\Q \text{ and } \sqrt{r}\in\Q \text{ then } r^{1/3}\in\Q."
\]

\vspace{1in}

Solution:  We also know this to be false.  Consider $r=4$.  Then $4\in\Q$ and $\sqrt{4} = 2\in\Q$, but the cube root is not a rational number.  Let's prove that:

\[
\text{Claim: } 4^{1/3}\notin \Q
\]

\begin{proof}
Let's prove this by contradiction.  Assume that $4^{1/3}\in\Q$.  Then there must be a pair of integers $a,b$ in lowest terms, so that
\[
4^{1/3} = \frac{a}{b} \text{ where } b\ne 0.
\]


Let's cube both sides and we see
\[
4 = \frac{a^3}{b^3}
\]

Now 
\[
a^3 = 4b^3 = 2(2b^3) \implies a^3\text{ is even } \implies a \text{ is even.}
\]

Since $a$ is even there is some $j\in\Z$ so that $a=2j$.  Now we see

\[
a^3 = (2j)^3 = 8j^3 = 4b^3 \implies b^3 = 2j^3 \implies b \text{ is even}.
\]

But $a$ and $b$ are in lowest terms and are both even.  This is a contradiction, and so $4^{1/3}$ is irrational.



\end{proof}



\newpage

Bonus: Let $p_1,p_2,p_3,p_4$ be prime numbers.  Is there a group of four prime numbers that satisfy the following equality?
\[
p_1^{p_2} - p_3^{p_4} = 1.
\]

(Aside from the obvious $3^2-2^3=1$.)

\end{document}