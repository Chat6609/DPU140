\documentclass[16 pt]{amsart}
\usepackage{amscd,amsmath,amsthm,amssymb}
\usepackage{enumerate,varioref}
\usepackage{epsfig}
\usepackage{graphicx}
\usepackage{mathtools}
\newtheorem{thm}{Theorem}
\newtheorem{cor}[thm]{Corollary}
\newtheorem{lem}[thm]{Lemma}
\newtheorem{prop}[thm]{Proposition}
\theoremstyle{definition}
\newtheorem{defn}[thm]{Definition}
\theoremstyle{remark}
\newtheorem{ex}[thm]{Example}
\newtheorem{rem}[thm]{Remark}
\numberwithin{equation}{subsection}
\newcommand{\R}{\mathbb{R}}
\newcommand{\Z}{\mathbb{Z}}
\newcommand{\C}{\mathbb{C}}
\newcommand{\Q}{\mathbb{Q}}
\newcommand{\lh}{\lim_{h\rightarrow 0}}
\begin{document}

\title{Final Exam Maths 140 Autumn 2014 \\ DePaul University\\Dr. Alexander}
\maketitle
You have 135 minutes to complete this exam.  Calculators are allowed, but no other electronic devices are permitted.  Please write all your answers in complete, legible sentences, and show all your work to receive full credit.  There are seven (9) problems here.  You may choose to do any eight (8) of them.  
\vspace{1in}


%table
\begin{center}
  \begin{tabular}{ c | c }
    Problem & Score\\
    \hline
    &\\
    1&\\
    &\\
    2&\\
    &\\
    3&\\
    &\\
    4&\\
    &\\
    5&\\
    &\\
    6&\\
    &\\
    7&\\
    &\\
    8&\\
    &\\
    9&\\
    &\\
    Bonus&\\
    &\\
    \hline 
    &\\    
    Total& 
 \end{tabular}
\end{center}

\newpage 
Problem 1. Give the proper negation of the statement:
\[
\forall x\in \Z, \text{ ``If $x^2$ is odd and $2x+1$ is prime then $x$ is prime."}
\]





\newpage
Problem 2. Suppose $a,b,c\in\Z$ and $x,y,z\in\R$ (and $x,y,z$ nonzero) satisfy the relations:
\[
\frac{xy}{x+y} = a, \hspace{.1in} \frac{xz}{x+z} = b, \hspace{.1in}\frac{yz}{y+z} = c
\]  

Is $x\in\Q$?  If so express $x$ as a ratio of two integers. If not, give a counterexample.




\newpage

Problem 3. Prove the following statement or give a counterexample: The product of two consecutive integers is even.

\newpage

Problem 4. The following matrix is an adjacency matrix for a graph.\\
\[
A = 
\begin{bmatrix}
0 & 1 & 1 & 1 & 1\\
1 & 0 & 1 & 1 & 0\\
1 & 1 & 0 & 1 & 0\\
1 & 1 & 1 & 0 & 0\\
1 & 0 & 0 & 0 & 0
\end{bmatrix}
\]
\\

(a) Draw the corresponding graph $G$.\\

(b) How many walks of length 2 are there from vertex one to vertex 3?


\newpage

Problem 5. Draw four graphs:\\
(a) Draw a graph with an Euler(ian) circuit, but not a Hamiltonian circuit.\\
(b) Draw a graph with a Hamiltonian circuit, but not an Euler(ian) circuit.\\
(c) Draw a (single) graph with both a Hamiltonian circuit and an Euler(ian) circuit.\\
(d) Draw a graph with neither a Hamiltonian nor Euler(ian) circuit.


\newpage

Problem 6. Define the function $f: \R \rightarrow \R$ by
\[
f(x) = \frac{x}{x^2+1}
\]

(a) Is $f$ injective (one to one)?\\

(b) Is $f$ surjective (onto)?


\newpage

Problem 7. What is the $33^{rd}$ element of the one dimensional array
\[
A[54],A[55],A[56],...,A[123]?
\]

\newpage

Problem 8. Suppose we pick 52 random integers.  Prove that the sum or difference of at least one pair must be a multiple of 100.


\newpage


Problem 9. How many numbers between 1 and 100 are divisible by 3,4, or 5?


\newpage
Bonus: What is the coefficient of $x^{100}$ in
\[
(x+x^2+x^5+x^{10}+x^{20}+x^{50})^{10}
\]

This is equivalent to asking how many ways we can make \$100 out of ten bills with lower denominations.

\end{document}