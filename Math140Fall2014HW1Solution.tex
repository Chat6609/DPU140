\documentclass[10 pt]{amsart}
\usepackage{amscd,amsmath,amsthm,amssymb}
\usepackage{enumerate,varioref}
\usepackage{epsfig}
\usepackage{graphicx}
\usepackage{mathtools}
\usepackage{xcolor}
\newtheorem{thm}{Theorem}
\newtheorem{cor}[thm]{Corollary}
\newtheorem{lem}[thm]{Lemma}
\newtheorem{prop}[thm]{Proposition}
\theoremstyle{definition}
\newtheorem{defn}[thm]{Definition}
\theoremstyle{remark}
\newtheorem{ex}[thm]{Example}
\newtheorem{rem}[thm]{Remark}
\numberwithin{equation}{subsection}
\newcommand{\R}{\mathbb{R}}
\newcommand{\Z}{\mathbb{Z}}
\newcommand{\C}{\mathbb{C}}
\newcommand{\Q}{\mathbb{Q}}
\newcommand{\lh}{\lim_{h\rightarrow 0}}
\begin{document}

\title{Homework 1 Maths 140 Autumn 2014}
\maketitle

Section 2.1.40: Use truth tables to establish which of the following are tautologies and which are contradictions\\
\[
(p\wedge q) \vee (\sim p \vee (p \wedge \sim q))
\]

Solution:\\
Let's first look at the method of brute force.  We'll write the whole truth table and see that we in fact have a tautology.

\[\begin{array}{c | c | c | c | c | c}
p & q & \sim p &  p\wedge q & p \wedge \sim q & (p\wedge q) \vee (\sim p \vee (p \wedge \sim q))\\
\hline
T & T & F & T & F & T \\
T & F & F & F & T & T \\
F & T & T & F & F & T \\
F & F & T & F & F & T 
\end{array}
\]

The second solution which is similar, but in some sense easier is to simply use the rules of our logical calculus and move around our symbols to combine them to a simpler statement.\\

Recall that $\vee$ is an associative operation which means
\[
(p\vee q)\vee r \equiv p \vee (q\vee r)
\]

So our original statement is
\[
(p\wedge q) \vee (\sim p \vee (p \wedge \sim q)) \equiv ((p\wedge q) \vee (p\wedge \sim q)) \vee \sim p
\]

Also recall that $\wedge$ dsitributes over $\vee$ and vice verse so that
\[
(p\wedge q) \vee (p\wedge \sim q) \equiv p \wedge (q \vee \sim q)
\]

We also know that 
\[
q\vee \sim q \equiv \mathfrak{t}
\] is a tautology and $p\wedge \mathfrak{t} \equiv p$.

So our original statement reduces in this way
\begin{eqnarray*}
(p\wedge q) \vee (\sim p \vee (p \wedge \sim q)) & \equiv & ((p\wedge q) \vee (p\wedge \sim q)) \vee \sim p \\
& \equiv & (p\wedge (q\vee\sim q))\vee \sim p\\
& \equiv & (p\wedge \mathfrak{t}) \vee \sim p\\
& \equiv & p \vee \sim p\\
& \equiv & \mathfrak{t} 
\end{eqnarray*}

\newpage

Section 2.1.42: Use truth tables to establish which of the following are tautologies and which are contradictions
\[
((\sim p \wedge q)\wedge (q\wedge r) \wedge \sim q
\]

Solution:\\
Again, let's first do the truth table and then reduce via our logical calculus.
\[
\begin{array}{c | c | c | c | c | c }
p & q & r & \sim p \wedge q & q\wedge r & ((\sim p \wedge q)\wedge (q\wedge r) \wedge \sim q\\
\hline
T & T & T & F & T & F\\
T & T & F & F & F & F\\
T & F & T & F & F & F\\
T & F & F & F & F & F\\
F & T & T & T & T & F\\
F & T & F & T & F & F\\
F & F & T & F & F & F\\
F & F & F & F & F & F
\end{array}
\]

We can clearly see that this is a contradiction. Now let's recall that $\wedge$ is also associative and simply reorder our variables.

\begin{eqnarray*}
(\sim p \wedge q)\wedge (q\wedge r) \wedge \sim q & \equiv & \sim p \wedge (q\wedge q) \wedge (r\wedge \sim q)\\
& \equiv & (\sim p \wedge q)\wedge (\sim q \wedge r)\\
& \equiv & (\sim p \wedge r) \wedge (q\wedge\sim q)\\
& \equiv & (\sim p \wedge r) \wedge \mathfrak{c}\\
& \equiv & \mathfrak{c}
\end{eqnarray*}

Since we have several $\wedge$ in a row and two contradictory variables, namely $q$ and $\sim q$ we are easily led to a contradiction.

\newpage

Section 2.2.15: Determine whether the following statements are logically equivalent
\[
p \rightarrow (q\rightarrow r) \text{ and } (p\rightarrow q)\rightarrow r
\]

Solution:\\
These two statements are not equivalent.  All that we must do to show this is find two values in the truth table which do not match.

\[
\begin{array}{c | c | c | c | c}
p & q & r & p \rightarrow (q\rightarrow r) & (p\rightarrow q) \rightarrow r\\
\hline
T & T & T & T & T\\
T & T & F & F & F\\
T & F & T & T & T\\
T & F & F & T & T\\
F & T & T & T & T\\
\hline
F & T & F & \color{red}{T} & \color{red}{F}\\
\hline
F & F & T & T & T\\
\hline
F & F & F & \color{red}{T} & \color{red}{F}\\
\hline
\end{array}
\]

So we see in rows 6 and 8 that the truth values don't match, thus these statements are not equivalent.


\newpage

Section 2.2.29: If the statement forms $P$ and $Q$ are logically equivalent, then $P\leftrightarrow Q$ is a tautology.  Conversely if $P\leftrightarrow Q$ is a tautology then $P$ and $Q$ are logically equivalent.  Use $\leftrightarrow$ to convert the following equivalence into a tautology.  Then use a truth table to verify the tautology.

\[
p\rightarrow (q\vee r) \equiv (p\wedge \sim q) \rightarrow r
\]

Solution:\\ 
Let's simply write the truth table
For the sake of saving a little typing space I'll call the statements 
\[
S_1 = p \rightarrow (q\vee r)
\]

 and

\[
S_2 = (p\wedge \sim q) \rightarrow r.
\]

Then our truth table is slightly more succint:

\[
\begin{array}{c | c | c | c | c | c}
p & q & r & S_1 & S_2 & S_1 \leftrightarrow S_2\\
\hline
T & T & T & T & T & T\\
T & T & F & T & T & T\\
T & F & T & T & T & T\\
T & F & F & F & F & T\\
F & T & T & T & T & T\\
F & T & F & T & T & T\\
F & F & T & T & T & T\\
F & F & F & T & T & T\\ 
\end{array}
\]


\newpage

Section 2.3.43: In the following problem a set of premises and a conclusion are given.  Use the valid argument forms we have discussed to deduce the conclusion from the premises, giving a reason for each step.  Assume all variables are statement variables.\\

\begin{enumerate}
\item[a.] $\sim p \rightarrow r \wedge \sim s$\\
\item[b.] $t \rightarrow s$\\
\item[c.] $u\rightarrow \sim p$\\
\item[d.] $\sim w$\\
\item[e.] $u\vee w$\\
\item[f.] $\therefore \sim t$\\
\end{enumerate}


Solution:\\
Let's put (d) and (e) together to see by elimination
\begin{eqnarray*}
u\vee w\\
\sim w\\
\therefore u
\end{eqnarray*}

Now that we know $u$.  Let's put this together with (c) and by modus ponens get $\sim p$.

\begin{eqnarray*}
u \rightarrow \sim p &\\
u &\text{ by previous argument}\\
\therefore \sim p &
\end{eqnarray*}

Again modus ponens with (a) tells us $r\wedge \sim s$
\begin{eqnarray*}
\sim p\rightarrow r\wedge \sim s\\
\sim p\\
\therefore r\wedge \sim s
\end{eqnarray*}

By specialization we know $\sim s$.
\begin{eqnarray*}
r\wedge \sim s\\
\therefore \sim s
\end{eqnarray*}

Combining this with (b) and modus tollens gives us the desired reult of $\sim t$
\begin{eqnarray*}
t\rightarrow s\\
\sim s\\
\therefore \sim t
\end{eqnarray*}

\end{document}