\documentclass[10 pt]{amsart}
\usepackage{amscd,amsmath,amsthm,amssymb}
\usepackage{enumerate,varioref}
\usepackage{epsfig}
\usepackage{graphicx}
\usepackage{mathtools}
\newtheorem{thm}{Theorem}
\newtheorem{cor}[thm]{Corollary}
\newtheorem{lem}[thm]{Lemma}
\newtheorem{prop}[thm]{Proposition}
\theoremstyle{definition}
\newtheorem{defn}[thm]{Definition}
\theoremstyle{remark}
\newtheorem{ex}[thm]{Example}
\newtheorem{rem}[thm]{Remark}
\numberwithin{equation}{subsection}
\newcommand{\R}{\mathbb{R}}
\newcommand{\Z}{\mathbb{Z}}
\newcommand{\C}{\mathbb{C}}
\newcommand{\Q}{\mathbb{Q}}
\newcommand{\lh}{\lim_{h\rightarrow 0}}
\begin{document}

\title{Homework 3 Maths 140 Autumn 2014}
\maketitle

Section 4.1.50: Determine whether the following statement is true or false.  If true, prove it.  If false, give a counterexample.\\

``For all integers $n$ and $m$, if $n-m$ is even then $n^3-m^3$ is also even."

Solution:\\
Here we have several valid forms of solution.  The first is to directly factor a difference of cubes.  The second is to prove this in cases.\\
First Solution:  We can recall that 
\[
n^3-m^3 = (n-m)(n^2+nm+m^2)
\]
If you can't remember this factorization, we can do it from scratch.  Realizing that $n^3-m^3$ has a zero when $n=m$ and thus $n-m$ is a factor.  So we have a first order polynomial and we need to solve for the second order polynomial.
\begin{eqnarray*}
n^3-m^3 & = & (n-m)(an^2 + bnm + cm^2)\\
        & = & an^3 + bn^2m + cnm^2 - an^2m - bnm^2 - cm^3\\
        & = & a n^3 + (b-a)n^2m + (c-b) nm^2- cm^3
\end{eqnarray*}
We can see clearly that $a=1$ and $c=1$ and $a=b$ and $b=c$.  So the factorization is as above
\[
n^3-m^3 = (n-m)(n^2+nm+m^2).
\]
So our proof goes as follows:
\begin{proof}
Since $n-m$ is even $\exists k\in\Z$ so that $n-m=2k$.  We have
\[
n^3-m^3 = (n-m)(n^2+nm+m^2) = 2k(n^2+nm+m^2)
\]
We also know that $n^2+nm+m^2$ is an integer because integers are closed under addition and multiplication.  Call it $\ell$. So we know
\[
n^3-m^3 = 2k(n^2+nm+m^2) = 2k\ell = 2(k\ell)
\]
Which is the definition of even.
\end{proof}


The second proof is slightly longer, but much simpler.  
\begin{proof}
We know that a difference of integers is even if, and only if the integers have the same parity, that is both are even or both are odd.  So let's check the first case, when both $n$ and $m$ are even.  This means there exist two integers $k$ and $\ell$ so that $n=2k$ and $m=2\ell$.  Then
\[
n^3-m^3 = (2k)^3 - (2\ell)^3 = 8k^3 - 8\ell^3 = 2(4k^2-4\ell^2)
\]
We know that $4k^2 -4\ell^2$ is an integer and so $n^3-m^3$ is even.\\
Now let's check the second case in which both $n$ and $m$ are odd.  So there are integers $i,j$ so that $n=2i+1$ and $m=2j+1$ Thus
\[
n^3-m^3 = (2i+1)^3 - (2j+1)^3 = 2r.
\]
For the sake of brevity we have skipped the expansion of $(2i+1)$ and $(2j+1)$.  In this case we know
\[
r = 4(i^3-j^3) + 6(i^2-j^2) + 3(i-j)
\]
And again $n^3-m^3$ is even.
\end{proof}


\newpage

Section 4.1.62: If $p$ is prime, must $2^p -1$ be prime?  Prove or give a counterexample.\\

Solution:\\
This is false.  The first counterexample is when $p=11$.  Since $2^{11} - 1 = 2047 = 23*89$ which is obviously composite.\\
In fact primes of the form $2^p-1$ are called Mersenne primes, and there are only 48 known.  In fact the vast majority of numbers of the form $2^p-1$ are composite.  However, the ``Great Internet Mersenne Prime Search" also known as ``GIMPS" is a massively parallelized search procedure which tries to find primes of this form.  In January 2013 the largest known prime was found via GIMPS and is more than 17 million digits long.

\newpage

Section 4.2.28: Suppose $a,b,c,d$ are integers and that $a\neq c$.  Suppose also that $x$ is a real number and satisfies the equation
\[
\frac{ax+b}{cx+d} =1.
\]
Must $x$ be rational?  If so, express $x$ as a ratio of two integers.\\


Solution:\\
Here we only need to solve for $x$ and then check that the resulting number is rational.

\begin{eqnarray*}
\frac{ax+b}{cx+d} & = & 1\\
ax+b & = & cx+d\\
ax-cx & = & d-b\\
x & = & \frac{d-b}{a-c}
\end{eqnarray*}

We know $d-b$ and $a-c$ are integers because integers are closed under subtraction.  We also know that $a-c \neq 0$ because we specified that $a\neq c$.


\newpage

Section 4.2.31: Prove that if a real number $c$ satisfies a polynomial of the form
\[
r_3 x^3 + r_2 x^2 + r_1 x + r_0 = 0
\]
where $r_i$ is rational of $0\leq i \leq 3$
Then $c$ also satisfies a polynomial of the form
\[
n_3 x^3 + n_2 x^2 + n_1 x+ n_0 = 0
\]
Where $n_i$ are integers for $0\leq i \leq 3$.\\

Solution:\\
To begin the solution here we need to look at an example that we already know how to do.  Consider for example
\[
\frac{1}{3} x^2 + \frac{1}{5}x + \frac{2}{3} = 0
\]
Our goal is to ``get rid of the denominator" which means we find the least common multiple and multiply both sides.  In this particular example the least common multiple is 15.  So the polynomial becomes
\[
5x^2 + 3x + 10 = 0.
\]
These are the same polynomials and therefore have the same roots.\\

In the general cubic case when we have four rational coefficients let's name $r_j = \frac{a_j}{b_j}$ and so we have
\[
r_3x^3 + r_2x^2 + r_1x + r_0 =0
\]
or with our newly renamed coefficients
\[
\frac{a_3}{b_3}x^3 + \frac{a_2}{b_2}x^2 + \frac{a_1}{b_1}x + \frac{a_0}{b_0} = 0
\]
The least common multiple (in general) will be a product of the four denominators.  So we have
\[
(b_3b_2b_1b_0)\left(\frac{a_3}{b_3}x^3 + \frac{a_2}{b_2}x^2 + \frac{a_1}{b_1}x + \frac{a_0}{b_0}\right) =(b_3b_2b_1b_0) 0
\]
Which transforms into
\[
(a_3b_2b_1b_0)x^3 + (b_3a_2b_1b_0)x^2 + (b_3b_2a_1b_0)x + (b_3b_2b_1a_0) = 0
\]
These are the same polynomials and thus have the same roots.  The only difference is that the final polynomial has integer coefficients.

\end{document}