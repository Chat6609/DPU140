\documentclass[16 pt]{amsart}
\usepackage{amscd,amsmath,amsthm,amssymb}
\usepackage{enumerate,varioref}
\usepackage{epsfig}
\usepackage{graphicx}
\usepackage{mathtools}
\newtheorem{thm}{Theorem}
\newtheorem{cor}[thm]{Corollary}
\newtheorem{lem}[thm]{Lemma}
\newtheorem{prop}[thm]{Proposition}
\theoremstyle{definition}
\newtheorem{defn}[thm]{Definition}
\theoremstyle{remark}
\newtheorem{ex}[thm]{Example}
\newtheorem{rem}[thm]{Remark}
\numberwithin{equation}{subsection}
\newcommand{\R}{\mathbb{R}}
\newcommand{\Z}{\mathbb{Z}}
\newcommand{\C}{\mathbb{C}}
\newcommand{\Q}{\mathbb{Q}}
\newcommand{\lh}{\lim_{h\rightarrow 0}}
\begin{document}

\title{Homework 4 Maths 140 Winter 2017 \\ DePaul University\\Dr. Alexander}
\maketitle

\section{Introduction}
In this problem set we'll begin exploring set theory and we'll get a hold on how to use some new notations.


\section{Problems}

Problem 1. Let $\{A_n\}_{n=1}^{\infty}$ be a sequence of sets defined by
\[
A_n = \{j\in\Z | 1\le j\le n\}
\]
That is $\forall n\in\mathbb{N}, |A_n|=n$.  Compute the following quantities:\\

(a) 
\[
\bigcup_{n=1}^{\infty} A_n
\]

(b) 
\[
\bigcap_{n=1}^{\infty} A_n 
\]

(c) Using only the operations $\cup,\cap,-$ and the sets $A_n$ how do we combine these sets to obtain
\[
\text{Some combination of sets }A_n = \{n\}
\]

(d) Using some combination of sets with the above operations, how to we achieve the sets
\[
\{1,3,5,\dots,2n+1\} \text{ i.e. the first } n \text{ odd positive integers.}
\]

\newpage

Problem 2.  Define the sequences of sets $\{A_n\}_{n\in\Z}$ and $\{B_n\}_{n\in\Z}$ by
\[
A_n = (n,n+1) \text{ the piece of the real numbers between two integers, not including the endpoints.}
\]


\[
B_n = [n,n+1] \text{ the piece of the real numbers between two integers, including the endpoints.}
\]

Compute the following quantities:\\

(a) 
\[
\bigcup_{n\in\Z} A_n
\]

(b)
\[
\bigcup_{n\in\Z} B_n
\]

(c) How can we combine these sets $A_n$ and $B_n$ to achieve the integers
\[
\text{Some combination of sets} = \Z 
\]

\newpage

Problem 3. Define the symmetric difference $\Delta$ of two sets $A,B$ by
\[
A\Delta B = (A-B)\cup(B-A) 
\]
The logical equivalent of this is ``exclusive or."\\


(a) Show that the above definition is equivalent to
\[
A\Delta B = (A\cup B)-(A\cap B)
\]

(b) Show that $\Delta$ is associative
\[
A\Delta (B\Delta C) = (A\Delta B)\Delta C
\]

(c) Give the Venn diagram for $A\Delta B\Delta C$.\\
Hint: Be careful, there may be an extra piece that you are forgetting.  For additional help, try computing the truth table for 
$p \text{ xor } q \text{ xor } r.$\\

(d) Let $A,B,C$ be three sets.   Prove the following:
\[
\forall C, \text{If } A\Delta C = B\Delta C \text{ then } A=B.
\]
Hint: You can prove this directly, but the contrapositive and division into cases is incredibly more efficient.

\newpage

Problem 4.  Give the set theoretic and logical equivalent proofs (two proofs per part) of the following:\\

(a)
\[
(A\cup B)\cap C = (A\cap C)\cup(B\cap C)
\]

(b)
\[
A\cup(B\cap C) = (A\cup B)\cap(A\cup C)
\]

(c) 
\[
\text{If } A\subseteq B \text{ then } A\cap C \subseteq B\cap C
\]

(d) We haven't seen the logical analog of the Cartesian product, so for this part, only give the set theoretic proof.  Let $A$ be a set and $B_n$ be some sequence of sets.
\[
A\times (\bigcap_{j=1}^n B_n) = \bigcap_{j=1}^n (A\times B_n)
\]

\newpage 

Problem 5. A Partition of a set $A$ is a set of subsets $\{A_i\}_{i=1}^n$ that satisfy two properties:

\begin{itemize}
\item $A_i \cap A_j = \emptyset$ if $i\ne j$\\
\item $\bigcup_{i=1}^n A_i = A$
\end{itemize}

(a) Consider the set $B=\{0,1,2\}$  How many partitions of $B$ are possible?\\

(b) Prove that the power set of a set can be partitioned by size.  That is, consider the sets:
\[
A_n = \{S\subseteq A | |S|=n\}
\]
Show these create a partition of $\mathcal{P}(A)$.



\newpage 

\section{Notes}


In problem three, the associativity lets us know that
\[
A\Delta B \Delta C
\]
is a well defined quantity.  In a similar way to addition or multiplication $a+b+c$ and $a*b*c$ both make sense.  However, as we have seen in the first homework
\[
(a^b)^c \ne a^{(b^c)}
\]
So that exponentiation is not associative, but as we saw earlier only right associative.

\par In problem five we start seeing the rudiments of a field called topology.  In lay person speech this is sometimes called ``rubber sheet geometry."  In topology we define what it means for a set to be ``open" by defining a partition of a set.  Such a set is called a topology on a set $A$.  For example, the standard topology on the real numbers that we use is to define the open sets
\[
(a,b) = \{x\in\R | a<x<b\}
\]
This, of course, gives us far more sets than necessary, but we can take a subset of these sets and ``cover" the real line.  We can't always create a partition with the open sets, but each open set gives us a partition of the real line by
\[
(a,b) \text{ and } (a,b)^c = (\infty,a]\cup[b,\infty)
\]




\end{document}