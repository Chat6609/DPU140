\documentclass[16 pt]{amsart}
\usepackage{amscd,amsmath,amsthm,amssymb}
\usepackage{enumerate,varioref}
\usepackage{epsfig}
\usepackage{graphicx}
\usepackage{mathtools}
\newtheorem{thm}{Theorem}
\newtheorem{cor}[thm]{Corollary}
\newtheorem{lem}[thm]{Lemma}
\newtheorem{prop}[thm]{Proposition}
\theoremstyle{definition}
\newtheorem{defn}[thm]{Definition}
\theoremstyle{remark}
\newtheorem{ex}[thm]{Example}
\newtheorem{rem}[thm]{Remark}
\numberwithin{equation}{subsection}
\newcommand{\R}{\mathbb{R}}
\newcommand{\Z}{\mathbb{Z}}
\newcommand{\C}{\mathbb{C}}
\newcommand{\Q}{\mathbb{Q}}
\newcommand{\lh}{\lim_{h\rightarrow 0}}
\begin{document}

\title{Takehome Assignment 2 Maths 140 Winter 2016 \\ DePaul University\\Dr. Alexander}
\maketitle
This assignment is due 1 March 2016.  Please complete all problems.  You may use all available resources online or amongst classmates or the grader.  You may not, however, ask tutors to do your work for you.  Each person must submit his or her own original work with proper citations.
\vspace{1in}


%table
\begin{center}
  \begin{tabular}{ c | c }
    Problem & Score\\
    \hline
    &\\
    1&\\
    &\\
    2&\\
    &\\
    3&\\
    &\\
    4&\\
    &\\
    5&\\
    &\\
    Bonus&\\
    &\\
    \hline 
    &\\    
    Total& 
 \end{tabular}
\end{center}

\newpage 
Problem 1. This graph is called the Petersen graph.  It is very important for many reasons.
\begin{center}
\includegraphics[scale=.5]{PetersenGraph}
\end{center}

(a) Show that this graph has neither an Euler(ian) circuit nor a Hamilton(ian) circuit.

\vspace{.25in}

(b) Show that is we remove any single vertex (and its corresponding edges) that the new graph now has a Hamilton(ian) circuit.

\vspace{.25in}

(c) Show that the Petersen graph is tripartite.  This is like bipartite, but the vertices are separated into three distinct classes.

\vspace{.25in}

(d) How many walk are there of length 5 from any vertex to itself?


\newpage

Problem 2. The $n$-hypercube is a graph $C(n)$ in $n$ dimensions given by the sets

\[
V(C(n)) = \{(a_1,a_2,\dots, a_n) | a_i \in \{0,1\} \}
\]

and two vertices are connected by an edge if their coordinates are exactly the same except at one spot.  For example, the square $C(2)$ is
\[
V = \{(0,0),(0,1),(1,0),(1,1)\}
\]

with edges
\[
\{\{(0,0),(0,1)\}, \{(0,0),(1,0)\},\{(1,1),(1,0)\},\{(1,1),(0,1)\}\} 
\]

And the normal cube likewise with eight vertices and 12 edges.

\vspace{.25in}

(a) How many vertices and edges are in $C(n)$? There is a pattern, so I want the completely general answer.  I should be able to ask you about $C(1234)$ and you should be able to tell me how many vertices and edges. 

\vspace{.25in}

(b) Draw $C(4)$ in two different ways.

\vspace{.25in}

(c) Which of these graph have Euler(ian) circuits? Which have Hamilton(ian) circuits?



\newpage

Problem 3. Give the appropriate conditions for a cyclic graph to be a subgraph of a complete bipartite graph.  For example, we know that $C_3$ is not a subgraph of any $K_{n,m}$.  However, $C_4 \simeq K_{2,2}$.  Additionally, $K_{2,2} \subseteq K_{n,m}$ for any $n,m \ge 2$.  So there are a few conditions.  What conditions allow us to say whether

\[
C_j \subseteq K_{n,m}?
\]

\vspace{.25in}

Hint: You'll want to use the Pigeonhole Principle several times here.

\newpage

Problem 4. Give examples of two $4\times 4$ matrices whose products satisfy

\vspace{.25in}

(a) $AB = I$ but neither $A$ nor $B$ is $I$.

\vspace{.25in}

(b) $AB = 0$ but no entries in $A$ or $B$ are zero.

\vspace{.25in}

(c) $AB=0$ but $BA \ne 0$.

\vspace{.25in}

(d) 
\[
AB - BA = \begin{bmatrix}
1 & 0 & 0 & 0\\
0 & 1 & 0 & 0\\
0 & 0 & -1 & 0\\
0 & 0 & 0 & -1
\end{bmatrix}
\]

\newpage

Problem 5. A secret santa gift exchange is one in which each participant brings a gift to exchange, and no person can go home with his or her own gift. How many possible secret santas are there in a party with 10 people?\\
Bonus for this question:  How many secret santas are possible with $n$ people?  

\vspace{.25in}

Hint 1: You will want to use the principle of inclusion/exclusion liberally here.\\
Hint 2: This is a fact from calculus that you may or may not know, but it can be very helpful:
\[
1 + \frac{1}{1!} + \frac{1}{2!} + \frac{1}{3!} + \cdots + \frac{1}{n!} \approx e \approx 2.718
\]


\newpage

Bonus: Consider the subset of the plane in which we only have integer pairs of points.  This is called the integer lattice.
\[
\Z \times \Z = \{(a,b) | a,b\in\Z \}
\]

Now consider the approximate circles or radius $r$ in the integer lattice, that is:
\[
C_r = \{(x,y)\in \Z\times\Z | x^2+y^2 \le r\}
\]

If each point in the integer lattice counts as one unit of area, what radius do you need to approximate $\pi$ to four decimal places?  That is if your approximation is called $\pi(r)$ what is the appropriate $r$ so that
\[
|\pi(r) - \pi | < 0.00001?
\]


\end{document}